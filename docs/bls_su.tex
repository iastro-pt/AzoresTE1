\documentclass[12pt,twoside]{article}
\usepackage{amssymb,amsmath,mathrsfs}
\usepackage[authoryear]{natbib}
\usepackage{float,graphicx}

%\usepackage{mparhack}
%\usepackage{numberparagraphs}
% \newcommand{\numberparagraphs}{}
%\newcommand{\nonumberparagraphs}{}

%%Figure caption
\makeatletter
\newsavebox{\tempbox}
\newcommand{\@makefigcaption}[2]{%
\vspace{10pt}{#1.--- #2\par}}%
\renewcommand{\figure}{\let\@makecaption\@makefigcaption\@float{figure}}
\makeatother

\newcommand{\exampleplot}[1]{%
\begin{center}%
\includegraphics[width=0.5\textwidth]{#1}%
\end{center}%
}
\newcommand{\exampleplottwo}[2]{%
\begin{center}%
\includegraphics[width=0.5\textwidth]{#1}%
\includegraphics[width=0.5\textwidth]{#2}%
\end{center}%
}


\setlength{\emergencystretch}{2em}%No overflow

\newcommand{\notenglish}[1]{\textsl{#1}}
\newcommand{\aposteriori}{\notenglish{a~posteriori}}
\newcommand{\apriori}{\notenglish{a~priori}}
\newcommand{\adhoc}{\notenglish{ad~hoc}}
\newcommand{\etal}{\notenglish{et al.}}
\newcommand{\eg}{\notenglish{e.g.}}

\newcommand{\documentname}{document}
\newcommand{\sectionname}{Section}
\newcommand{\equationname}{equation}
\newcommand{\figurenames}{\figurename s}
\newcommand{\problemname}{Exercise}
\newcommand{\problemnames}{\problemname s}
\newcommand{\solutionname}{Solution}
\newcommand{\notename}{note}

\newcommand{\note}[1]{\endnote{#1}}
\def\enotesize{\normalsize}
\renewcommand{\thefootnote}{\fnsymbol{footnote}} % the ONE footnote needs this

\newcounter{problem}
\newenvironment{problem}{\paragraph{\problemname~\theproblem:}\refstepcounter{problem}}{}
\newcommand{\affil}[1]{{\footnotesize\textsl{#1}}}


% other random multiply used math symbols
\renewcommand{\d}{\mathrm{d}}
\newcommand{\like}{\mathscr{L}}
\newcommand{\pfg}{p_{\mathrm{fg}}}
\newcommand{\pbg}{p_{\mathrm{bg}}}
\newcommand{\Pbad}{P_{\mathrm{b}}}
\newcommand{\Ybad}{Y_{\mathrm{b}}}
\newcommand{\Vbad}{V_{\mathrm{b}}}
\newcommand{\bperp}{b_{\perp}}
\newcommand{\mean}[1]{\left<{#1}\right>}
\newcommand{\meanZ}{\mean{\mZ}}

% header stuff
\usepackage{fancyhdr}
\usepackage{lipsum} % only for showing some sample text
\fancyhf{} % clear all header and footers
\renewcommand{\headrulewidth}{0pt} % remove the header rule
\rfoot{\thepage}
%
%lfoot{\thepage} % puts it on the left side instead
%
% or if your document is 2 sided, and you want even and odd placement of the number
%\fancyfoot[LE,RO]{\thepage} % Left side on Even pages; Right side on Odd pages
%
\pagestyle{fancy}
\renewcommand{\MakeUppercase}[1]{#1}
% \pagestyle{myheadings}
\renewcommand{\sectionmark}[1]{\markright{\thesection.~#1}}
\markboth{}{}

\begin{document}


\section*{Transit search}

% \begin{quote}
    % For the transit search we will use the most common method called the Box-fitting Least Squares algorithm \citep{Kovacs2002}.
% \end{quote}

The methods used to estimate periodicities in time series data can be divided in two main types: Fourier analysis 
%(Deeming 1975; Lomb 1976; Scargle 1982; van der Klis 1988; Press \& Rybicki 1989; Press et al. 1992; Gray 1992; Bracewell 1986) 
and folding techniques. 
%(Stellingwerf 1978; Leahy et al. 1983; Davies 1990, 1991). 
Fourier analysis methods are based in the decomposition of the signal into sinusoidal functions of a given frequency, that is, the calculation of Fourier transforms (FT). They are useful when we want to detect a sinusoidal signal, even in low signal-to-noise data. Folding techniques consist in folding the data over trial periods and then analysing the scatter of the resulting profile with an appropriate statistic.
%Using, for example, the epoch-folding statistic we can obtain the probability of the existence of a periodicity as a function of the trial period. 
Folding techniques perform better than FTs for non-sinusoidal periodic signals, but can fail in the presence of multiple periods.

% When searching for periodic signals in stellar light curves it is common to use the Discrete Fourier Transform or for the more general case of  not evenly sampled data the lomb scargle periodogram.
When searching for transit signals in stellar light curves, the signal is (extremely) non-sinusoidal and localised in time. A large number of harmonics is needed to describe the function, which can lead to leakage of the power to higher harmonics. To avoid this problem, \citet{Kovacs2002} proposed the Box-fitting Least Squares (BLS) algorithm that uses box-shaped functions to fit the signal instead of sinusoids. The box shape is the superposition of two step functions with opposite signs representing a low and a high state with the low state lasting much less time.
 
The algorithm works as a folding technique, calculating the power for a defined set of frequencies with frequency separation \textbf{df}, a minimum frequency \textbf{fmin} and the number of frequency bins to test \textbf{nf}. The light curve is folded at each trial frequency and binned in phase into \textbf{nb} bins. Then the algorithm searches for the low state based within a fractional duration range ( \textbf{qmi-qma}) and then fits the  \textbf{depth},  \textbf{duration} and the start \textbf{in1} and end \textbf{in2} of the transit, with the power corresponding to the goodness of fit. The result is a periodogram (power as a function of each trial frequency or period), and estimation of the fitted parameters for the best period.
 



\vfill
\begin{problem}\label{prob:first}
Search for transits in the K2 C4 light curve of EPIC 211089792, using a \texttt{Python} implementation of the BLS algorithm.
Search for periods in the range from 0.5 days to 70\% of the full duration of the observations, with a frequency resolution of 0.001 and choose the remaining  parameters nb, qmi, qma.
% Calculate the BLS periodogram for these frequencies and plot it. \\
Plot the resulting BLS periodogram.
Identify the frequency of highest peak and its corresponding parameters: power, depth, duration, epoch.
Phase fold the light curve using the period found. How does it look?
Try out different input parameters and discuss the best result.
Take note of the derived parameters.
\end{problem}


% \nocite{Hogg2010a}
% \renewcommand{\refname}{\normalfont\selectfont\normalsize \textbf{Based on}}
\bibliographystyle{plainnat}
\bibliography{/home/joao/phd/bib/zotero_library}

\end{document}