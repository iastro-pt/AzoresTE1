\documentclass[12pt,twoside]{article}
\usepackage{amssymb,amsmath,mathrsfs,natbib}
\usepackage{float,graphicx}
\usepackage{hyperref}

%\usepackage{mparhack}
%\usepackage{numberparagraphs}
% \newcommand{\numberparagraphs}{}
%\newcommand{\nonumberparagraphs}{}

%%Figure caption
\makeatletter
\newsavebox{\tempbox}
\newcommand{\@makefigcaption}[2]{%
\vspace{10pt}{#1.--- #2\par}}%
\renewcommand{\figure}{\let\@makecaption\@makefigcaption\@float{figure}}
\makeatother

\newcommand{\exampleplot}[1]{%
\begin{center}%
\includegraphics[width=0.5\textwidth]{#1}%
\end{center}%
}
\newcommand{\exampleplottwo}[2]{%
\begin{center}%
\includegraphics[width=0.5\textwidth]{#1}%
\includegraphics[width=0.5\textwidth]{#2}%
\end{center}%
}


\setlength{\emergencystretch}{2em}%No overflow

\newcommand{\notenglish}[1]{\textsl{#1}}
\newcommand{\aposteriori}{\notenglish{a~posteriori}}
\newcommand{\apriori}{\notenglish{a~priori}}
\newcommand{\adhoc}{\notenglish{ad~hoc}}
\newcommand{\etal}{\notenglish{et al.}}
\newcommand{\eg}{\notenglish{e.g.}}

\newcommand{\documentname}{document}
\newcommand{\sectionname}{Section}
\newcommand{\equationname}{equation}
\newcommand{\figurenames}{\figurename s}
\newcommand{\problemname}{Exercise}
\newcommand{\problemnames}{\problemname s}
\newcommand{\solutionname}{Solution}
\newcommand{\notename}{note}

\newcommand{\note}[1]{\endnote{#1}}
\def\enotesize{\normalsize}
\renewcommand{\thefootnote}{\fnsymbol{footnote}} % the ONE footnote needs this

\newcounter{problem}
\newenvironment{problem}{\paragraph{\problemname~\theproblem:}\refstepcounter{problem}}{}
\newcommand{\affil}[1]{{\footnotesize\textsl{#1}}}

% parameter vectors
\newcommand{\parametervector}[1]{\mmatrix{#1}}
\newcommand{\pvtheta}{\parametervector{\theta}}


% other random multiply used math symbols
\renewcommand{\d}{\mathrm{d}}
\newcommand{\like}{\mathscr{L}}
\newcommand{\pfg}{p_{\mathrm{fg}}}
\newcommand{\pbg}{p_{\mathrm{bg}}}
\newcommand{\Pbad}{P_{\mathrm{b}}}
\newcommand{\Ybad}{Y_{\mathrm{b}}}
\newcommand{\Vbad}{V_{\mathrm{b}}}
\newcommand{\bperp}{b_{\perp}}
\newcommand{\mean}[1]{\left<{#1}\right>}
\newcommand{\meanZ}{\mean{\mZ}}

% header stuff
\usepackage{fancyhdr}
\usepackage{lipsum} % only for showing some sample text
\fancyhf{} % clear all header and footers
\renewcommand{\headrulewidth}{0pt} % remove the header rule
\rfoot{\thepage}
%
%lfoot{\thepage} % puts it on the left side instead
%
% or if your document is 2 sided, and you want even and odd placement of the number
%\fancyfoot[LE,RO]{\thepage} % Left side on Even pages; Right side on Odd pages
%
\pagestyle{fancy}
\renewcommand{\MakeUppercase}[1]{#1}
% \pagestyle{myheadings}
\renewcommand{\sectionmark}[1]{\markright{\thesection.~#1}}
\markboth{}{}

\begin{document}


\section*{The (generalized) Lomb-Scargle periodogram}

% \begin{quote}
%     This problem sheet is based on \ldots
% \end{quote}


The Lomb-Scargle periodogram is commonly used for period search and frequency analysis of time series. It is equivalent to fitting sine waves of the form $y = a \cos\omega t+b \sin \omega t$, for a given frequency $\omega$. This method provides an analytic solution and is therefore both convenient to use and efficient. The equation for the periodogram was given by \cite{Barning1963}, and \cite{Lomb1976} and \cite{Scargle1982} further investigated its statistical behaviour,
especially the statistical significance of the detection of a signal.

Because the detection of periodicities in time series data is such a common problem, many generalisations of the Lomb-Scargle periodogram have been proposed. These take into account individual measurement errors, the inclusion of a constant offset, or other types of periodic signals \citep{Ferraz-Mello1981,Cumming1999,Zechmeister2009a,Mortier2015}.
Computer implementations of these algorithms are also common and mature in many programming languages.

\vfill
\begin{problem}\label{prob:easy}
Use the \texttt{astropy} implementation of the Lomb-Scargle periodogram to search for periodic variations in the radial velocities of EPIC 211089792. Take into account each individual RV uncertainty. Plot the periodogram, and calculate the period of maximum power. Fold the RV observations at the best period.
\end{problem}





\begin{problem}\label{prob:standard}
\end{problem}





\begin{problem}\label{prob:quadratic}
\end{problem}

% \nocite{Hogg2010a}
% \renewcommand{\refname}{\normalfont\selectfont\normalsize \textbf{Based on}}
\bibliographystyle{plainnat}
\bibliography{/home/joao/phd/bib/zotero_library}

\end{document}